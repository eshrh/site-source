\documentclass[10pt, letterpaper]{moderncv}
\usepackage[utf8]{inputenc}
\usepackage[top=2cm, bottom=2cm]{geometry}
\usepackage{CJKutf8}
\moderncvtheme{classic}
\moderncvcolor{custom}

\firstname{Eshan}
\familyname{Ramesh}

\email{esrh@gatech.edu}
\homepage{esrh.me}

\renewcommand{\emailsymbol}{}
\renewcommand{\homepagesymbol}{}
\begin{document}
\makecvtitle

\section{Education}
\cventry{2021-Present}{B.S Computer Science}{Georgia Institute of Technology}{Atlanta, GA}
{Anticipated graduation in 2024}{}

\section{Experience}
\cventry{2022-2023}{Intern}{NTT Network Innovation
  lab \begin{CJK}{UTF8}{ipxm}(未来ねっと研究所)\end{CJK}, Frontier Communications group}{Yokosuka, Japan}{}
{Utilizing physical information to make inferences via machine learning about wireless conditions.}

\cventry{2022 (summer)}{Researcher}{Communications architectures
  research group}{Atlanta, GA}{}
{Worked on single-access-point active localization with software
  defined radio. Work sponsored by NSF REU. \\
Technology: GNUradio, USRP, Python, NumPy}

\cventry{2021-2022}{Researcher}{Agile communications and architectures vertically integrated project}{Atlanta, GA}{}
{Worked on the passive-localization/activity detection problem with
  software-defined radio.\\
Technology: GNUradio, Software Defined Radios(USRP), Python, Tensorflow, NumPy.}

\cventry{2020}{Researcher}{Florida State University Young Scholars Program}{Tallahassee, FL}{}
{Analyzed plankton biomass data from National Ecological Observatory Network.
\\Technology: Python, Tensorflow, NumPy, Pandas, Matplotlib, Scikit-learn.}

\subsection{Projects}
\cvline{\httplink[py-NEONutils]{https://gitlab.com/esrh/py-neonutils}}
    {Libary to download and organize NEON(National Ecological Observatory Network) data. Used Python and Pandas.}
\cvline{\httplink[AMES]{https://github.com/eshrh/ames}}
    {Generate audiovisual context for spaced-repetition flashcards automatically. Written in shell.}

\section{Computer skills}
\cvline{Languages}{\textbf{Python, Java, C++, Bash, Lisp dialects (Elisp, Clojure, Common Lisp)}}
\cvline{Technology}{\textbf{GNU/Linux, Git, \LaTeX, Emacs}}
\cvline{Libraries}{\textbf{Tensorflow, NumPy, Pandas, Matplotlib, Scikit-learn}}

\section{Natural Languages}
\cvlanguage{English}{Native}{}
\cvlanguage{Kannada}{Native}{}
\cvlanguage{Japanese}{Fluent}{Comprehensible input approach}
\cvlanguage{French}{Beginner}{}
\end{document}
