\documentclass[11pt, a4paper, sans]{moderncv}
\usepackage[utf8]{inputenc}
\usepackage[top=2cm, bottom=2cm]{geometry}
\usepackage{hyperref}
\usepackage{verbatim}
\usepackage{CJKutf8}
\moderncvtheme{classic}
\moderncvcolor{cerulean}

\firstname{Eshan}
\familyname{Ramesh}

\email{esrh@esrh.me}
\homepage{esrh.me}

\renewcommand{\emailsymbol}{}
\renewcommand{\homepagesymbol}{}

\AtBeginDocument{
  \hypersetup{
    breaklinks=false,
    baseurl={https://},
    pdfborder={0 0 1},
    allbordercolors={color1},
  }
}

\begin{document}
\makecvtitle

\section{Education}
\cventry{2024-Present}{M.S Information and Communications Engineering}
{Tokyo Institute of Technology}{Tokyo, Japan} {}{}

\cventry{2021-2024 [2yr]}{B.S Computer Science}
{Georgia Institute of Technology}{Atlanta, GA} {
  Specialization in AI and CS theory
}{}

\section{Experience}
\cventry{2024-Present}{Research Assistant}{Nishio Laboratory}{Tokyo,
  Japan}{} {
  Generating camera images from WiFi CSI. Developed a new fast
  latent-space method, and built a practical, scalable, distributed
  prototype.\\
  \textbf{[Linux WiFi firmware/networking, Rust, Python ML]}
  \begin{itemize}
    \item High-resolution efficient image generation from
      WiFi CSI using a pretrained latent diffusion model
      \href{https://esrh.me/files/pub3.pdf}{(IEEE Globecom
      2025/11)}
    \item LatentCSI demo proposal
      \href{https://esrh.me/files/pub4_demo.pdf}{(ACM Mobicom 2025/12)}
  \end{itemize}
}

\cventry{2022-2023}{Intern}
{NTT Network Innovation lab \begin{CJK}{UTF8}{ipxm}(未来ねっと研究所)
  \end{CJK}}
{Yokosuka, Japan}{}{
  Utilized physical information to make inferences about wireless
  conditions with machine learning. Worked on an automatic data generation
  testbed with moving humanoid robots carrying commerical 5G terminals. \\
  \textbf{[Panasonic 5G, ROS (Lisp; Python), LiDAR, Python ML]}
  \begin{itemize}
    \item 5G throughput prediction using LiDAR
      \href{https://esrh.me/files/pub1.pdf}{(IEICE SeMI 2023/5)}
    \item 5G throughput
      prediction in 28GHz cells using physical information
      \href{https://esrh.me/files/pub2.pdf}{(Japanese; IEICE SeMI 2023/7)}
  \end{itemize}
}

\cventry{2021-2022}{Research Assistant}
{Communications architectures group}{Atlanta, GA}{}{
  Worked on single-access-point active localization with software
  defined radio. Low level radio hacking. Work sponsored by NSF.\\
\textbf{[GNUradio (C++), NI USRP, Python ML]}}

\subsection{Projects {\small \href{https://github.com/eshrh}{\texttt{[gh:eshrh]}}}}

\cvline{\httplink[inori]{https://github.com/eshrh/inori}}
{TUI music player [Rust, 80 stars]}

\cvline{\httplink[ames]{https://github.com/eshrh/ames}}
{Generates audiovisual context for spaced-repetition flashcards automatically. [POSIX shell, 70 stars]}

\section{Computer skills}
\cvline{Languages}{\textbf{Python, Shell, Lisp dialects (Clojure, Common Lisp, Racket), C++}}
\cvline{Technology}{\textbf{GNU/Linux, SDR, Git, \LaTeX, Emacs}}

\section{Natural Languages}
\cvlanguage{Native}{English \cvskill{5}}{}
\cvlanguage{Proficient}{Japanese \cvskill{4}, Kannada \cvskill {3}}{}
\cvlanguage{Basic}{Mandarin \cvskill{1}, French \cvskill{1}}{}
\end{document}
